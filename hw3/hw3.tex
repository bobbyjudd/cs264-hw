\documentclass{article}

% If you're new to LaTeX, here's some short tutorials:
% https://www.overleaf.com/learn/latex/Learn_LaTeX_in_30_minutes
% https://en.wikibooks.org/wiki/LaTeX/Basics

% Formatting
\usepackage[utf8]{inputenc}
\usepackage[margin=1in]{geometry}

% Math
% https://www.overleaf.com/learn/latex/Mathematical_expressions
% https://en.wikibooks.org/wiki/LaTeX/Mathematics
\usepackage{amsmath,amsfonts,amssymb,mathtools}

% Images
% https://www.overleaf.com/learn/latex/Inserting_Images
% https://en.wikibooks.org/wiki/LaTeX/Floats,_Figures_and_Captions
\usepackage{graphicx,float}

% Tables
% https://www.overleaf.com/learn/latex/Tables
% https://en.wikibooks.org/wiki/LaTeX/Tables

% Algorithms
% https://www.overleaf.com/learn/latex/algorithms
% https://en.wikibooks.org/wiki/LaTeX/Algorithms
\usepackage[ruled,vlined]{algorithm2e}
\usepackage{algorithmic}

% Code syntax highlighting
% https://www.overleaf.com/learn/latex/Code_Highlighting_with_minted
\usepackage{minted}
\usemintedstyle{borland}


\usepackage{tikz}
\usetikzlibrary{positioning}

% Title content
\title{CS264A Homework 2}
\author{Bobby Judd}
\date{November 4th, 2020}

\begin{document}

\maketitle

% 1
\clearpage
\section{}
The amount of different $k-1$ combinations out of $n$ items would be ${n \choose k-1}$.  For each of those unique combinations any number of the items $A_i$ could be selected, including none of them and it would still satisfy the constraint.  Conversely, all of the remaining $n-(k-1)$ item for each combination MUST NOT be selected. If $(A_k, A_{k+1}, ... , A_{n-1}, A_{n})$ represents the set of items not currently selected then the CNF to represent this looks like:
\[
\Delta = \lnot A_k \land \lnot A_{k+1} \land ... \land \lnot A_{n-1}\land \lnot A_{n}
\]

% 2 
\clearpage
\section{}
Converting the CNF $\Delta$ to DNF and reducing produces the boolean formula:
\[f = (A \land \lnot B) \lor (A \land \lnot C) \]


\renewcommand{\labelenumi}{(\alph{enumi})}
 \begin{enumerate}
   \item The partition of $X = \{A,B\}$ and $Y = \{C\}$ produces:
   \begin{center}
           \begin{tabular}{ |c|c| }
            \hline
             Prime&Sub \\ 
             \hline
             $A \land B$ & $\lnot C$ \\
             \hline
             $A \land \lnot B$ & $true$ \\
             \hline
             $\lnot A \land B$ & $false$ \\
             \hline
             $\lnot A \land \lnot B$ & $false$ \\
             \hline
            \end{tabular} \\
    \end{center}
    The compression of the partition is:
    \begin{center}
       \begin{tabular}{ |c|c| }
        \hline
         Prime&Sub \\ 
         \hline
         $A \land B$ & $\lnot C$ \\
         \hline
         $A \land \lnot B$ & $true$ \\
         \hline
         $\lnot A$ & $false$ \\
         \hline
        \end{tabular} \\
    \end{center}
        
   \item To construct an SDD you would want to use the vtree (a) because the left subtree contains \{A, B\} and the right subtree contains \{C\} 
 \end{enumerate}

% 3
\clearpage
\section{}
\renewcommand{\labelenumi}{(\alph{enumi})}
 \begin{enumerate}
\item The partition of $f = (A \land B)\lor(B \land C)\lor(C \land D)$ for $X = \{A,C\}$ and $Y = \{B, D\}$ produces:
\begin{center}
       \begin{tabular}{ |c|c| }
        \hline
         Prime&Sub \\ 
         \hline
         $A \land C$ & $B \lor D$ \\
         \hline
         $A \land \lnot C$ & $B$ \\
         \hline
         $\lnot A \land C$ & $B \lor D$ \\
         \hline
         $\lnot A \land \lnot C$ & $false$ \\
         \hline
        \end{tabular} \\
\end{center}
The compression of the partition is:
\begin{center}
       \begin{tabular}{ |c|c| }
        \hline
         Prime&Sub \\ 
         \hline
         $A \land \lnot C$ & $B$ \\
         \hline
         $C$ & $B \lor D$ \\
         \hline
         $\lnot A \land \lnot C$ & $false$ \\
         \hline
        \end{tabular} \\
\end{center}
    
\item The partition of $\lnot f$ for $X = \{A,C\}$ and $Y = \{B, D\}$ produces:
\begin{center}
       \begin{tabular}{ |c|c| }
        \hline
         Prime&Sub \\ 
         \hline
         $A \land C$ & $\lnot B \land \lnot D$ \\
         \hline
         $A \land \lnot C$ & $\lnot B$ \\
         \hline
         $\lnot A \land C$ & $\lnot B \land \lnot D$ \\
         \hline
         $\lnot A \land \lnot C$ & $true$ \\
         \hline
        \end{tabular} \\
\end{center}
The compression of the partition is:
\begin{center}
       \begin{tabular}{ |c|c| }
        \hline
         Prime&Sub \\ 
         \hline
         $A \land \lnot C$ & $\lnot B$ \\
         \hline
         $C$ & $\lnot B \land \lnot D$ \\
         \hline
         $\lnot A \land \lnot C$ & $true$ \\
         \hline
        \end{tabular} \\
\end{center}

\item For any function $f$ you can simply copy its primes and negate its subs to produce the partition of $\lnot f$
\end{enumerate}
   
 % 4
 \clearpage
 \section{}
 
 % 5
 \clearpage
 \section{}

 % 6
 \clearpage
 \section{}
 
 % 7
 \clearpage
 \section{}

% 8
\clearpage
\section{}

% 9
\clearpage
\section{}

% 10
\clearpage
\section{}

\end{document}